\section{יחידה 4: מדדי מיקום יחסי ומדדי קשר}

יחידה זו עוסקת במיקום של תצפית ביחס להתפלגות כולה, ובהמשך בקשר בין שני משתנים.
הדגש הוא על תקנון (ציוני תקן), אחוזונים, ומתאם פירסון.

\subsection{מדדי מיקום יחסי – רעיון כללי}

מדדי מיקום יחסי עונים על השאלה:
\begin{quote}
איפה התצפית נמצאת ביחס לשאר התצפיות?
\end{quote}

בניגוד למדדי מרכז ופיזור, הם מתארים \textbf{תצפית בודדת} ולא את ההתפלגות כולה.



\subsection{ציוני תקן (Z-score)}

\subsubsection{הגדרה}
\textbf{ציון תקן} מתאר בכמה סטיות תקן תצפית רחוקה מהממוצע:
\[
Z_i = \frac{x_i - \bar{X}}{S_n}
\]

\subsubsection{הסבר אינטואיטיבי}
כמה “חריג” הערך ביחס לשאר המדגם, תוך התחשבות בפיזור.

\subsubsection{דגשים למבחן}
\begin{itemize}
\item $Z>0$ – תצפית מעל הממוצע.
\item $Z<0$ – תצפית מתחת לממוצע.
\item $Z=0$ – תצפית שווה לממוצע.
\item ציון תקן הוא \textbf{חסר יחידות}.
\end{itemize}



\subsection{למה משתמשים בציוני תקן}

\begin{itemize}
\item השוואת תצפיות מאותו משתנה במדגמים שונים.
\item השוואת אותה תכונה שנמדדה בכלים שונים.
\item השוואת תצפיות ממשתנים שונים לחלוטין (גובה, משקל, ציונים).
\end{itemize}

\subsubsection{דגש למבחן}
השוואה בין ציונים גולמיים \textbf{אינה} מספיקה — יש להשוות ציוני תקן.



\subsection{תכונות מדגם לאחר תקנון}

לאחר המרה לציוני תקן:
\begin{itemize}
\item ממוצע ציוני התקן: $\bar{Z} = 0$
\item סטיית תקן ציוני התקן: $S_Z = 1$
\item צורת ההתפלגות נשמרת.
\end{itemize}

\subsubsection{דגש למבחן}
תקנון \textbf{לא משנה} את עוצמת הקשר בין משתנים.



\subsection{מצבים מיוחדים בציוני תקן}

\begin{itemize}
\item אם $S_n = 0$ (כל הערכים זהים) — \textbf{לא ניתן לחשב ציון תקן}.
\item ייתכן:
  \begin{itemize}
  \item ציון תקן שלילי אך ערך מעל החציון.
  \item אחוזון גבוה עם $Z$ שלילי (בהתפלגות מוטה).
  \end{itemize}
\end{itemize}



\subsection{אחוזונים}

\subsubsection{הגדרה}
האחוזון ה-$p$ הוא ערך שמתחתיו נמצאים $p\%$ מהתצפיות.

\subsubsection{הסבר אינטואיטיבי}
האחוזון אומר \textbf{כמה אחוזים נמצאים מתחת לערך}, לא מעליו.

\subsubsection{דגשים למבחן}
\begin{itemize}
\item האחוזון ה־50 הוא החציון.
\item אין צורך לדעת את צורת ההתפלגות כדי לפרש אחוזון.
\end{itemize}

\subsubsection{טעות נפוצה}
לחשוב שאחוזון 80 אומר שהערך גדול מ־80\% מהטווח — לא נכון.



\subsection{מדדי קשר – מבוא}

מדדי קשר עונים על השאלה:
\begin{quote}
האם מידע על משתנה אחד מספק מידע על משתנה אחר?
\end{quote}

קשר ≠ סיבתיות.


\subsection{דיאגרמת פיזור \texorpdfstring{\textenglish{(Scatter Plot)}}{(Scatter Plot)}}
\subsubsection{תיאור}
גרף שבו כל תצפית מיוצגת כנקודה במישור $(X,Y)$.

\subsubsection{מה בודקים}
\begin{itemize}
\item כיוון הקשר: חיובי / שלילי.
\item עוצמת הקשר: חזק / חלש / אין קשר.
\item קיצוניים והשפעתם.
\end{itemize}


\subsection{מקדם המתאם של פירסון}

\subsubsection{תנאים}
\begin{itemize}
\item שני משתנים בסולם רווח או יחס.
\item קשר לינארי.
\end{itemize}

\subsubsection{הגדרה}
מקדם המתאם של פירסון מודד את עוצמת וכיוון הקשר הלינארי:
\[
-1 \le r \le 1
\]

\subsubsection{פירוש}
\begin{itemize}
\item $r>0$ – קשר חיובי.
\item $r<0$ – קשר שלילי.
\item $r=0$ – אין קשר לינארי.
\item $|r|$ קרוב ל־1 -> קשר חזק.
\end{itemize}

מתאם פירסון $(r)$ מודד קשר לינארי בין שני משתנים כמותיים.
\begin{figure}[H]
\centering
\begin{english}
\begin{tikzpicture}[scale=0.8]
  \draw[->] (0,0) -- (4,0) node[right] {$X$};
  \draw[->] (0,0) -- (0,4) node[above] {$Y$};
  \foreach \p in {(0.5,0.7), (1,1.5), (1.8,2.1), (2.5,2.8), (3.2,3.5)}
    \fill[blue] \p circle (2.5pt);
  \draw[red, dashed] (0,0) -- (4,4);
\end{tikzpicture}
\end{english}
\caption{דיאגרמת פיזור - קשר חיובי.}
\end{figure}

\subsection{נוסחת מקדם המתאם}

\[
r = \frac{\sum_{i=1}^{n} (x_i - \bar{x})(y_i - \bar{y})}{n \cdot S_{nx} \cdot S_{ny}}
\]

או שקול:
\[
r = \frac{1}{n}\sum Z_{x_i} Z_{y_i}
\]

\subsubsection{דגש למבחן}
מתאם הוא \textbf{קו-וריאנס מתוקנן}.



\subsection{תכונות חשובות של מתאם פירסון}

\begin{itemize}
\item חסר יחידות.
\item סימטרי: $r(x,y)=r(y,x)$.
\item הוספת קבוע או כפל בקבוע חיובי — לא משנה את $r$.
\item כפל בקבוע שלילי — הופך את סימן $r$.
\end{itemize}



\subsection{מצבים שבהם מתאם אינו מוגדר}

\begin{itemize}
\item סטיית התקן של אחד המשתנים שווה לאפס.
\end{itemize}

\subsubsection{דגש למבחן}
לא אומרים $r=0$ — אלא \textbf{לא מוגדר}.

\subsection{קשר חזק \texorpdfstring{$\neq$}{שונה מ-} סיבתיות}

גם אם $r$ גבוה:
\begin{itemize}
\item לא ניתן להסיק על השפעה סיבתית.
\item ייתכן משתנה מתערב.
\end{itemize}

\subsection{השפעת טרנספורמציות לינאריות על מדדים יחסיים}

נבחן טרנספורמציה מהצורה:
\[
Y = aX + b
\]

\subsubsection{השפעה על ציון תקן $(Z)$}

\begin{itemize}
\item הוספת קבוע ($b \neq 0$):  
ציון התקן \textbf{אינו משתנה}.
\item הכפלה בקבוע חיובי ($a>0$):  
ציון התקן \textbf{אינו משתנה}.
\item הכפלה בקבוע שלילי ($a<0$):  
סימן ציון התקן \textbf{מתהפך}.
\end{itemize}

\subsubsection{השפעה על מתאם פירסון $(r)$}

\begin{itemize}
\item הוספת קבוע לאחד המשתנים:  
המתאם \textbf{אינו משתנה}.
\item הכפלה בקבוע חיובי:  
המתאם \textbf{אינו משתנה}.
\item הכפלה בקבוע שלילי:  
סימן המתאם \textbf{מתהפך}.
\end{itemize}

\subsubsection{דגש למבחן}
טרנספורמציות לינאריות \textbf{אינן משנות עוצמת קשר},  
רק את כיוונו במקרה של כפל שלילי.

\subsection{טעויות נפוצות במבחנים}

\begin{itemize}
\item לחשב מתאם פירסון למשתנה מסולם סדר.
\item להסיק סיבתיות ממתאם.
\item לבלבל בין עוצמת הקשר לסימן הקשר.
\item לחשוב שתקנון משנה מתאם.
\end{itemize}

\subsection{הערת העמקה: תקנון ומתאם פירסון}

\subsubsection{למה לאחר תקנון הממוצע הוא 0}

נגדיר לכל תצפית:
\[
z_i = \frac{x_i - \bar{x}}{S_x}
\]

הממוצע של ציוני התקן הוא:
\[
\bar{z} = \frac{1}{n}\sum_{i=1}^{n} z_i
= \frac{1}{S_x}\cdot \frac{1}{n}\sum_{i=1}^{n}(x_i - \bar{x})
\]

מאחר ש:
\[
\sum_{i=1}^{n}(x_i - \bar{x}) = 0
\]

נקבל:
\[
\bar{z} = 0
\]



\subsubsection{למה סטיית התקן והשונות לאחר תקנון שוות ל־1}

השונות של ציוני התקן:
\[
S_z^2 = \frac{1}{n}\sum_{i=1}^{n} z_i^2
= \frac{1}{S_x^2}\cdot \frac{1}{n}\sum_{i=1}^{n}(x_i - \bar{x})^2
\]

מאחר ש:
\[
S_x^2 = \frac{1}{n}\sum_{i=1}^{n}(x_i - \bar{x})^2
\]

נקבל:
\[
S_z^2 = 1 \qquad \Rightarrow \qquad S_z = 1
\]



\subsubsection{למה מתאם פירסון תמיד בין \(-1\) ל-\(1\)}

ניתן לכתוב את מתאם פירסון כך:
\[
r = \frac{1}{n}\sum_{i=1}^{n} z_{x_i}z_{y_i}
\]

זהו ממוצע המכפלות של ציוני התקן, וניתן לפרשו כמכפלה סקלרית של שני וקטורים מתוקננים.
לפי אי־שוויון קושי–שוורץ (המבטא גבול גיאומטרי על מכפלה סקלרית של וקטורים),
מתקבל כי:
\[
|r| \le 1
\]


שוויון מתקבל כאשר הקשר הלינארי מושלם (חיובי או שלילי).

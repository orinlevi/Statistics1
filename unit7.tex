\section{יחידה 7: מעבר ממדגם לאוכלוסייה – סטטיסטים, פרמטרים ומשתנים מקריים}

יחידה זו עוסקת במעבר מהנתונים שנצפים בפועל (מדגם)  
אל מודל תאורטי של האוכלוסייה שממנה המדגם נלקח.  
זהו מעבר מושגי קריטי: משכיחויות והסתברויות אמפיריות  
אל הסתברות כתכונה של מודל.

\subsection{מדגם ואוכלוסייה}

\subsubsection{הגדרות}
\begin{itemize}
\item \textbf{אוכלוסייה (Population)} – אוסף כל המקרים האפשריים בעלי עניין, לרוב תאורטי ואינו נצפה במלואו.
\item \textbf{מדגם (Sample)} – תת־קבוצה סופית של האוכלוסייה שנמדדת בפועל.
\end{itemize}

\subsubsection{דגש מושגי}
\begin{itemize}
\item באוכלוסייה מדברים על \textbf{הסתברויות}.
\item במדגם מדברים על \textbf{שכיחויות}.
\item המדגם נועד לשמש מקור מידע על האוכלוסייה.
\end{itemize}



\subsection{פרמטר וסטטיסטי}

\subsubsection{הגדרות}
\begin{itemize}
\item \textbf{פרמטר (Parameter)} – גודל קבוע המתאר את האוכלוסייה כולה (למשל: $\mu, \sigma, p$).
\item \textbf{סטטיסטי (Statistic)} – גודל מחושב מתוך המדגם (למשל: $\bar{X}, S, \hat{p}$).
\end{itemize}

\subsubsection{דוגמאות}
\begin{itemize}
\item $\mu$ – ממוצע האוכלוסייה  
\item $\bar{X}$ – ממוצע המדגם  
\item $p$ – הסתברות באוכלוסייה  
\item $\hat{p}$ – שכיחות יחסית במדגם
\end{itemize}

\subsubsection{דגש למבחן}
סטטיסטי הוא \textbf{משתנה מקרי}.  
פרמטר הוא \textbf{קבוע לא ידוע}.



\subsection{\hebeng{משתנה מקרי}{Random Variable}}

\subsubsection{הגדרה}
משתנה מקרי הוא פונקציה המתאימה ערך מספרי לכל תוצאה אפשרית בניסוי מקרי.

\begin{english}
\[
X : \Omega \rightarrow \mathbb{R}
\]
\end{english}

\subsubsection{פירוש}
המשתנה המקרי אינו “המדידה עצמה” אלא ייצוג מתמטי של תוצאת הניסוי.


\subsection{משתנה מקרי בדיד ורציף}

\subsubsection{בדיד (Discrete)}
\begin{itemize}
\item מקבל מספר סופי או בן־מנייה של ערכים.
\item ניתן לדבר על:
\[
P(X = x)
\]
\end{itemize}

\subsubsection{רציף (Continuous)}
\begin{itemize}
\item מקבל אינסוף ערכים בתחום רציף.
\item ההסתברות לערך בודד היא:
\[
P(X = x) = 0
\]
\item הסתברות מוגדרת רק על קטעים.
\end{itemize}

\subsubsection{דגש למבחן}
האפס אינו אומר “בלתי אפשרי” –  
אלא “זניח יחסית לאינסוף האפשרויות”.


\subsection{שכיחות, שכיחות יחסית והסתברות}

\subsubsection{במדגם}
\begin{itemize}
\item שכיחות: מספר הפעמים שערך מופיע.
\item שכיחות יחסית:
\[
\frac{f_i}{n}
\]
\end{itemize}

\subsubsection{באוכלוסייה}
\begin{itemize}
\item אין ספירה בפועל.
\item יש הסתברות כתכונה של המודל.
\end{itemize}

\subsubsection{קשר חשוב}
כאשר גודל המדגם גדל:
\[
\text{שכיחות יחסית} \longrightarrow \text{הסתברות}
\]


\subsection{תוחלת (Expectation)}

\subsubsection{רעיון}
התוחלת היא הערך הממוצע של משתנה מקרי בטווח הארוך.

\subsubsection{בדיד}
\[
E(X) = \sum x \cdot P(X = x)
\]

\subsubsection{רציף}
\begin{english}
\[
E(X) = \int_{-\infty}^{\infty} x \cdot f(x)\,dx
\]
\end{english}

\subsubsection{פירוש אינטואיטיבי}
אם נבצע את הניסוי מספר רב מאוד של פעמים –  
הממוצע של התוצאות יתקרב לתוחלת.


\subsection{סטטיסטי כמשתנה מקרי}

\subsubsection{רעיון מרכזי}
סטטיסטי (כמו $\bar{X}$) תלוי במדגם – ולכן הוא משתנה מקרי.

\subsubsection{משמעות}
\begin{itemize}
\item לסטטיסטי יש התפלגות.
\item ניתן לדבר על תוחלת ושונות של סטטיסטי.
\end{itemize}


\subsection{אמידה (Estimation)}

\subsubsection{אומד (Estimator)}
סטטיסטי שנועד להעריך פרמטר לא ידוע.

\subsubsection{דוגמה}
\[
\bar{X} \text{ הוא אומד ל־ } \mu
\]


\subsection{תכונות של אומדים}

\subsubsection{חוסר הטיה (Unbiasedness)}
\[
E(\hat{\theta}) = \theta
\]

\subsubsection{יעילות (Efficiency)}
אומד עם שונות קטנה יותר עדיף.

\subsubsection{קונסיסטנטיות (Consistency)}
כאשר $n \to \infty$:
\[
\hat{\theta} \to \theta
\]

\subsubsection{מספיקות (Sufficiency)}
האומד מכיל את כל המידע הרלוונטי על הפרמטר.


\subsection{דגש מסכם ליחידה}

יחידה זו מבצעת את המעבר:
\[
\text{נתונים} \rightarrow \text{סטטיסטים}
\rightarrow \text{מודל הסתברותי}
\rightarrow \text{פרמטרים}
\]

זהו הבסיס לכל הסטטיסטיקה ההיסקית בהמשך הקורס.

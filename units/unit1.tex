\section{יחידה 1: מושגי יסוד וסולמות מדידה}

\noindent
דגש: אם מחלקה רחבה יותר, גובה העמודה חייב להתאים כדי שה\textbf{שטח} יישאר פרופורציונלי לשכיחות.

יחידה זו עוסקת במושגי היסוד של הסטטיסטיקה ומהווה בסיס לכל הקורס. הבנה מדויקת של סוגי משתנים, סולמות מדידה והטרנספורמציות המותרות עליהם היא תנאי לשימוש נכון בכלים סטטיסטיים בהמשך.

\subsection{משתנים}

\subsubsection{הגדרה}
\textbf{משתנה} הוא תכונה או תופעה שיכולה לקבל ערכים שונים בין נבדקים שונים או בין מדידות שונות.

\subsubsection{הסבר אינטואיטיבי}
משתנה הוא כל דבר שיכול להשתנות: מספר ילדים במשפחה, גובה אדם, רמת שביעות רצון, טמפרטורה, או מספר טעויות בביצוע מטלה.

\subsubsection{דגשים למבחן}
\begin{itemize}
\item יש להבחין בין \textbf{המשתנה עצמו} לבין \textbf{הערכים שהוא מקבל}.
\item משתנה יכול להיות מספרי או קטגוריאלי, אך לא כל מספרי הוא “כמותי” במובן הסטטיסטי.
\end{itemize}

\subsubsection{טעויות נפוצות}
\begin{itemize}
\item בלבול בין שם המשתנה (למשל: גובה) לבין יחידות המדידה (מטרים, סנטימטרים).
\end{itemize}



\subsection{משתנה בדיד ומשתנה רציף}

\subsubsection{הגדרה}
\begin{itemize}
\item \textbf{משתנה בדיד}: משתנה שערכיו ניתנים למנייה, ואין ערכים אפשריים בין שני ערכים עוקבים.
\item \textbf{משתנה רציף}: משתנה שיכול לקבל (תיאורטית) אינסוף ערכים בין כל שני ערכים.
\end{itemize}

\subsubsection{הסבר אינטואיטיבי}
במשתנה בדיד סופרים (ילדים, טעויות, זבובים).  
במשתנה רציף מודדים (זמן, אורך, משקל).

\subsubsection{דוגמאות}
\begin{itemize}
\item בדיד: מספר ילדים במשפחה, מספר טעויות במבחן.
\item רציף: גובה, משקל, זמן תגובה.
\end{itemize}

\subsubsection{דגשים למבחן}
\begin{itemize}
\item העובדה שהמדידה בפועל מעוגלת \textbf{לא הופכת} משתנה רציף לבדיד.
\item בדיד ≠ סולם שמי, רציף ≠ סולם יחס. אלו הבחנות שונות.
\end{itemize}

\subsubsection{טעויות נפוצות}
\begin{itemize}
\item לחשוב ש“כל מספרי הוא רציף”.
\end{itemize}



\subsection{סולמות מדידה}

\subsubsection{מדידה}
\textbf{מדידה} היא תהליך שבו מייחסים ערכים מספריים לתכונות או תופעות כך שהמספרים משקפים את היחסים ביניהן.

\subsubsection{סולם מדידה}
\textbf{סולם מדידה} מגדיר את המשמעות המתמטית של הערכים ולכן קובע אילו פעולות סטטיסטיות מותר לבצע עליהם.



\subsection{ארבעת סולמות המדידה}

\subsubsection{סולם שמי}
\textbf{הגדרה}:  
סולם שבו הערכים משמשים כתוויות זיהוי בלבד. אין סדר, אין רווחים ואין יחסים.

\textbf{דוגמאות}: מין, עיר מגורים, מספר חולצה.

\textbf{דגשים למבחן}:
\begin{itemize}
\item מותר רק לבדוק זהות או אי־זהות.
\item אין משמעות לגודל המספר.
\end{itemize}



\subsubsection{סולם סדר}
\textbf{הגדרה}:  
סולם שבו קיימת היררכיה בין הערכים, אך אין משמעות לרווחים ביניהם.

\textbf{דוגמאות}: דירוג בתחרות, רמת שביעות רצון.

\textbf{דגשים למבחן}:
\begin{itemize}
\item יודעים מי יותר ומי פחות, אך לא בכמה.
\end{itemize}

\subsubsection{טעות נפוצה}
להניח שאם יש סדר – יש גם רווחים שווים.



\subsubsection{סולם רווח}
\textbf{הגדרה}:  
סולם שבו יש משמעות לרווחים בין ערכים, אך אין משמעות ליחסים, ונקודת האפס נקבעת באופן שרירותי.

\textbf{דוגמאות}: טמפרטורה במעלות צלזיוס, שנים בלוח השנה.

\textbf{דגשים למבחן}:
\begin{itemize}
\item הבדל של 10 מעלות הוא זהה בכל מקום בסולם.
\item אין משמעות ל“פי שניים”.
\end{itemize}



\subsubsection{סולם יחס (מנה)}
\textbf{הגדרה}:  
סולם שבו יש משמעות לרווחים וליחסים בין ערכים, וקיים אפס מוחלט המייצג היעדר תכונה.

\textbf{דוגמאות}: גובה, משקל, מספר ילדים, טמפרטורה בקלווין.

\textbf{דגשים למבחן}:
\begin{itemize}
\item אפשר לומר “פי שניים”.
\item אפס = היעדר מוחלט של התכונה.
\end{itemize}



\subsection{היררכיה בין סולמות מדידה}

\textbf{היררכיה}:
\[
\text{שמי} \rightarrow \text{סדר} \rightarrow \text{רווח} \rightarrow \text{יחס}
\]

כל סולם גבוה מכיל את התכונות של הסולמות שמתחתיו.

\subsubsection{דגשים למבחן}
\begin{itemize}
\item לרוב נעדיף למדוד בסולם הגבוה ביותר האפשרי.
\item סולם המדידה תלוי \textbf{גם באופן המדידה}, לא רק בתכונה.
\end{itemize}



\subsection{טרנספורמציות על סולמות מדידה}

\subsubsection{הגדרה}
\textbf{טרנספורמציה} היא פעולה מתמטית המבוצעת על כל ערכי המשתנה.

\subsubsection{טרנספורמציות מותרות}
\begin{itemize}
\item שמי: כל טרנספורמציה ששומרת זהות.
\item סדר: כל טרנספורמציה ששומרת סדר.
\item רווח: טרנספורמציה לינארית $y = ax + b$, כאשר $a \neq 0$.
\item יחס: הכפלה בקבוע $y = ax$, כאשר $a \neq 0$.
\end{itemize}

\subsubsection{דגשים למבחן}
\begin{itemize}
\item הוספת קבוע שומרת רווח אך שוברת יחס.
\item הכפלה בקבוע שומרת יחס.
\end{itemize}



\subsection{טעויות נפוצות ובלבולים שכיחים}

\begin{itemize}
\item לחשוב שכל משתנה עם אפס הוא בהכרח סולם יחס.
\item לבלבל בין בדיד/רציף לבין סולם המדידה.
\item להניח שציונים במבחן הם תמיד סולם רווח (נחשב אזור אפור).
\end{itemize}


\section{יחידה 6: הסתברות}

יחידה זו עוסקת בהגדרה פורמלית של הסתברות, מאורעות ויחסים ביניהם.  
הדגש הוא על חשיבה נכונה: זיהוי מרחב המדגם, המאורע הרצוי, ושימוש בחוקי הסתברות בסיסיים.



\subsection{ניסוי מקרי ומרחב מדגם}

\subsubsection{הגדרות}
\begin{itemize}
\item \textbf{ניסוי מקרי}: תהליך שתוצאתו אינה ידועה מראש.
\item \textbf{מרחב מדגם} ($\Omega$): קבוצת כל התוצאות האפשריות של הניסוי.
\item \textbf{מאורע}: תת־קבוצה של מרחב המדגם.
\end{itemize}

\subsubsection{דוגמה}
בהטלת קובייה:
\[
\Omega = \{1,2,3,4,5,6\}
\]



\subsection{הסתברות של מאורע}

\subsubsection{הגדרה}
כאשר כל התוצאות שוות הסתברות:
\[
P(A) = \frac{|A|}{|\Omega|}
\]

\subsubsection{דגשים למבחן}
\begin{itemize}
\item תמיד לבדוק: מהו \textbf{הסה״כ}? מהו \textbf{הרצוי}?
\item $P(\Omega)=1$
\item $0 \leq P(A) \leq 1$
\end{itemize}



\subsection{מאורעות משלימים}

\subsubsection{הגדרה}
המשלים של מאורע $A$ מסומן $\bar{A}$ ומכיל את כל התוצאות שאינן ב־$A$.

\[
P(\bar{A}) = 1 - P(A)
\]

\subsubsection{דגש למבחן}
המילים \textbf{״לפחות״}, \textbf{״לא״}, \textbf{״אף אחד״} → לעצור ולחשוב על משלים.



\subsection{איחוד וחיתוך של מאורעות}

\subsubsection{איחוד}
\[
A \cup B = \{\,A \text{ או } B \text{ או שניהם}\,\}
\]

\subsubsection{חיתוך}
\[
A \cap B = \{\,A \text{ וגם } B\,\}
\]

\subsubsection{חוק האיחוד}
\[
P(A \cup B) = P(A) + P(B) - P(A \cap B)
\]

\subsubsection{דגשים למבחן}
\begin{itemize}
\item לשים לב לא לספור פעמיים את החיתוך.
\item אם המאורעות זרים → $P(A \cap B)=0$
\end{itemize}



\subsection{מאורעות זרים ומאורעות משלימים}

\subsubsection{מאורעות זרים}
\[
A \cap B = \emptyset
\]

\subsubsection{מאורעות משלימים}
\[
A \cup \bar{A} = \Omega
\]

\subsubsection{דגש למבחן}
מאורעות משלימים הם תמיד גם זרים וגם ממצים.

\textbf{דגש חשוב:}  
כל שני \textbf{מאורעות פשוטים שונים} הם מאורעות זרים,  
אך מאורע פשוט \textbf{אינו} זר לעצמו.



\subsection{הסתברות מותנית (ברמה אינטואיטיבית)}

\subsubsection{רעיון}
הסתברות של מאורע \textbf{בהינתן} שמאורע אחר כבר קרה.

\subsubsection{דוגמה}
מה הסיכוי שסטודנט אוכל בריא \textbf{בהינתן} שהוא עושה ספורט?

\subsubsection{דגש למבחן}
לעיתים אין צורך בנוסחה – מספיק לצמצם את מרחב המדגם.



\subsection{בעיות רצף וחזרות}

\subsubsection{רעיון}
כאשר מבצעים ניסוי חוזר (כמו הזמנת קולה):
\begin{itemize}
\item אם יש תלות → מכפילים הסתברויות
\item רצף מסוים → סדר חשוב
\end{itemize}

\subsubsection{דוגמה}
בדיוק 5 פעמים קולה בלי גזים מתוך 7:
\[
\binom{7}{5} \cdot 0.8^5 \cdot 0.2^2
\]

\subsubsection{דגש למבחן}
\begin{itemize}
    \item "בדיוק" \mbox{$\leftarrow$} קומבינציה
    \item "ברצף" \mbox{$\leftarrow$} בלי קומבינציה
\end{itemize}



\subsection{שיטת המשלים}

\subsubsection{רעיון}
כשקשה לחשב ישירות – מחשבים את המקרה היחיד שלא רוצים.

\[
P(\text{לפחות אחד}) = 1 - P(\text{אף אחד})
\]

\subsubsection{דוגמה}
לפחות עוגה טעימה אחת:
\[
1 - P(\text{שתיהן לא טעימות})
\]



\subsection{שילוב הסתברות וקומבינטוריקה}

\subsubsection{עיקרון}
\[
P = \frac{\text{רצוי}}{\text{סה״כ}}
\]

\subsubsection{דוגמאות קלאסיות}
\begin{itemize}
\item יד פוקר
\item בחירת קלפים
\item סידורי ישיבה
\end{itemize}



\subsection{טעויות נפוצות בהסתברות}

\begin{itemize}
\item לשכוח משלים
\item לבלבל בין איחוד לחיתוך
\item לא להגדיר נכון את מרחב המדגם
\item לחשוב ש״או״ הוא תמיד חיבור פשוט
\end{itemize}

\subsection{דגש מסכם למבחן}

לפני חישוב:
\begin{enumerate}
\item מהו הניסוי?
\item מהו מרחב המדגם?
\item מה הרצוי?
\item יש תלות? יש סדר?
\item שווה לבדוק משלים?
\end{enumerate}

\subsection{הסתברות מותנית}

\subsubsection{הגדרה}
ההסתברות של מאורע $A$ בהינתן שמאורע $B$ התרחש:
\[
P(A \mid B) = \frac{P(A \cap B)}{P(B)}, \qquad P(B)>0
\]

\subsubsection{הסבר אינטואיטיבי}
הסתברות מותנית היא הסתברות בעולם מצומצם:
\begin{itemize}
\item אנו יודעים ש־$B$ קרה
\item לכן מרחב המדגם מצטמצם ל־$B$
\item בודקים מה החלק של $A$ בתוך העולם הזה
\end{itemize}

\subsubsection{דגש למבחן}
לא מחשבים “סתם לפי הנוסחה” — קודם מצמצמים את מרחב המדגם בראש.



\subsection{הקשר בין חיתוך להסתברות מותנית}

מההגדרה מתקבל:
\[
P(A \cap B) = P(A \mid B)\cdot P(B)
\]

\subsubsection{דגש למבחן}
אם כתוב:
\begin{quote}
״בהינתן ש־$B$ קרה, מה ההסתברות שגם $A$ קרה?״
\end{quote}
סביר מאוד שצריך חיתוך דרך הסתברות מותנית.




\subsection{עצמאות בין מאורעות}

עצמאות היא מקרה פרטי חשוב של הסתברות מותנית, שבו הידיעה על מאורע אחד \textbf{אינה משנה} את ההסתברות של המאורע השני.

\subsubsection{הגדרה}
מאורעות $A$ ו־$B$ נקראים \textbf{בלתי־תלויים} אם:
\[
P(A \mid B) = P(A)
\]

\subsection{מאורעות תלויים}

\subsubsection{דוגמה מילולית קלאסית}
בדיקה רפואית:
\begin{itemize}
\item $A$ = האדם חולה
\item $B$ = תוצאת הבדיקה חיובית
\end{itemize}
ההסתברות להיות חולה \textbf{בהינתן} בדיקה חיובית שונה מההסתברות להיות חולה באופן כללי:
\[
P(A \mid B) \neq P(A)
\]
ולכן המאורעות תלויים.

\subsubsection{הגדרה}
מאורעות $A$ ו־$B$ נקראים \textbf{תלויים} אם ידיעת התרחשותו של אחד מהם משנה את ההסתברות של השני:
\[
P(A \mid B) \neq P(A)
\]

\subsubsection{הקשר לחיתוך}
במקרה של תלות, חישוב החיתוך מתבצע באמצעות הסתברות מותנית:
\[
P(A \cap B) = P(B)\cdot P(A \mid B)
\]
ואין להשתמש בנוסחה $P(A)\cdot P(B)$.

\subsubsection{דוגמה מספרית}
נניח:
\[
P(A)=0.1, \qquad P(B\mid A)=0.9, \qquad P(B\mid \bar{A})=0.2
\]
אז:
\[
P(B)=0.9\cdot 0.1 + 0.2\cdot 0.9 = 0.27
\]
ולכן:
\[
P(A\mid B)=\frac{P(B\mid A)P(A)}{P(B)}=\frac{0.9\cdot 0.1}{0.27}=\frac{1}{3}
\]
כיוון ש־$P(A\mid B)\neq P(A)$, המאורעות תלויים.

\subsubsection{טבלת השוואה}

\begin{center}
\begin{tabular}{|c|c|c|c|}
\hline
סוג מאורעות & תנאי פורמלי & חיתוך & דוגמה אינטואיטיבית \\
\hline
זרים & $A\cap B=\emptyset$ & $0$ & קובייה: 1 וגם 6 \\
\hline
בלתי־תלויים & $P(A\mid B)=P(A)$ & $P(A)P(B)$ & שתי הטלות מטבע \\
\hline
תלויים & $P(A\mid B)\neq P(A)$ & $P(B)\cdot P(A\mid B)$ & בדיקה רפואית \\
\hline
\end{tabular}
\end{center}

\subsubsection{דגש למבחן}
\begin{itemize}
\item לא כל מאורעות שאינם זרים הם עצמאיים.
\item אם הנתון כולל מילים כמו \textbf{״בהינתן״}, לרוב יש תלות.
\item תמיד לבדוק האם מותר להשתמש במכפלה פשוטה.
\end{itemize}


\subsubsection{דוגמה לעצמאות}
שתי הטלות של מטבע הוגן:
\begin{itemize}
\item $A$ = יצא עץ בהטלה הראשונה
\item $B$ = יצא עץ בהטלה השנייה
\end{itemize}
ההטלה השנייה אינה מושפעת מהראשונה, ולכן:
\[
P(A\mid B)=P(A)=\tfrac{1}{2}
\]
והמאורעות בלתי־תלויים.

\subsubsection{נוסחה שקולה}
\[
P(A \cap B) = P(A)\cdot P(B)
\]

\subsubsection{הסבר אינטואיטיבי}
התרחשות של $B$ לא משנה את הסיכוי של $A$.

\subsubsection{דגש למבחן}
עצמאות היא תכונה מתמטית — לא אינטואיטיבית.
אין להסיק עצמאות מתוך ניסוח מילולי בלבד.



\subsection{טעויות נפוצות בעצמאות}

\begin{itemize}
\item לבלבל בין מאורעות זרים לעצמאיים  
(מאורעות זרים עם הסתברות חיובית \textbf{אינם} עצמאיים)
\item להניח שעצמאות נובעת מ״אין קשר סיבתי״
\end{itemize}



\subsection{נוסחת בייס}

\subsubsection{הגדרה}
נוסחת בייס מתקבלת ישירות מהגדרת הסתברות מותנית:
\[
P(A \mid B) = \frac{P(B \mid A)\cdot P(A)}{P(B)}
\]

כאשר:
\[
P(B) = P(B \mid A)P(A) + P(B \mid \bar{A})P(\bar{A})
\]



\subsection{מתי צריך בייס}

\begin{itemize}
\item כשנותנים הסתברויות בכיוון אחד (אבחון, בדיקה)
\item ושואלים הסתברות בכיוון ההפוך
\end{itemize}

\subsubsection{דוגמה טיפוסית}
בדיקה רפואית:
\begin{itemize}
\item $P(\text{חיובי} \mid \text{חולה})$ נתון
\item שואלים $P(\text{חולה} \mid \text{חיובי})$
\end{itemize}



\subsection{עץ הסתברויות}

\subsubsection{רעיון}
עץ הסתברות מייצג ניסוי רב־שלבי:
\begin{itemize}
\item הסתברויות על הענפים
\item מכפילים לאורך מסלול
\item מחברים מסלולים לאותו מאורע
\end{itemize}

\subsubsection{דגש למבחן}
עץ טוב יכול לחסוך שימוש מפורש בנוסחת בייס.



\subsection{השוואה: בייס מול עץ}

\begin{itemize}
\item בייס — נוסחתי, אלגנטי, קצר
\item עץ — ויזואלי, אינטואיטיבי, בטוח מטעויות
\end{itemize}

בשניהם מתקבלת אותה תוצאה.



\subsection{טעויות נפוצות במבחנים}

\begin{itemize}
\item לבלבל בין $P(A \mid B)$ ל־$P(B \mid A)$
\item לשכוח לחשב את $P(B)$ במכנה של בייס
\item להניח עצמאות בלי בדיקה
\item לא לצמצם מרחב מדגם לפני חישוב
\end{itemize}



\subsection{דגש מסכם ליחידה}

לפני חישוב:
\begin{enumerate}
\item מה ידוע?
\item על מה מתנים?
\item האם המאורעות עצמאיים?
\item נוח יותר עץ או בייס?
\end{enumerate}

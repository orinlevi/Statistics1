\section{יחידה 2: מדדי מרכז}

מדדי נטייה מרכזית נועדו לתאר ערך מייצג של התפלגות. ביחידה זו נלמד שלושה מדדים עיקריים: שכיח, חציון וממוצע, ואת הקשר ביניהם לבין סולמות מדידה ופונקציות הפסד.

\subsection{מדדי נטייה מרכזית – מבט כללי}

שלושת מדדי הנטייה המרכזית הם:
\begin{itemize}
\item \textbf{שכיח} – הערך שמופיע בתדירות הגבוהה ביותר.
\item \textbf{חציון} – הערך האמצעי בהתפלגות מסודרת.
\item \textbf{ממוצע} – סכום הערכים חלקי מספר התצפיות.
\end{itemize}

לא כל מדד מתאים לכל משתנה, והבחירה ביניהם תלויה בסולם המדידה ובמטרת התיאור.



\subsection{השכיח}

\subsubsection{הגדרה}
\textbf{השכיח} הוא הערך שמופיע בשכיחות הגבוהה ביותר בהתפלגות.

\subsubsection{הסבר אינטואיטיבי}
זהו הערך ה”נפוץ ביותר” – מה שרואים הכי הרבה.

\subsubsection{דגשים למבחן}
\begin{itemize}
\item זהו מדד הנטייה המרכזית \textbf{היחיד} שמתאים לסולם שמי.
\item ייתכן יותר משכיח אחד (התפלגות רב־שכיחית).
\item ייתכן מצב שבו אין שכיח כלל.
\end{itemize}

\subsubsection{דוגמה קצרה}
אם מזון מועדף של קבוצה מתחלק לקטגוריות, המדד היחיד שניתן לחשב הוא השכיח.

\subsubsection{טעויות נפוצות}
\begin{itemize}
\item לחשוב שתמיד קיים שכיח.
\item לבלבל בין שכיח לערך “ממוצע” אינטואיטיבית.
\end{itemize}



\subsection{החציון}

\subsubsection{הגדרה}
\textbf{החציון} הוא הערך שמחלק את ההתפלגות לשני חצאים שווים, לאחר סידור הערכים לפי גודל.

\subsubsection{הסבר אינטואיטיבי}
חצי מהתצפיות קטנות ממנו וחצי גדולות ממנו.

\subsubsection{חישוב}
\begin{itemize}
\item מספר תצפיות אי־זוגי: החציון הוא הערך האמצעי.
\item מספר תצפיות זוגי: החציון הוא ממוצע שני הערכים האמצעיים.
\end{itemize}

\subsubsection{דגשים למבחן}
\begin{itemize}
\item החציון מתאים לפחות לסולם סדר.
\item החציון \textbf{עמיד לערכים קיצוניים}.
\item שינוי בערך קיצוני לא בהכרח ישפיע על החציון.
\end{itemize}

\subsubsection{דוגמה מבחנית}
אם מוסיפים להתפלגות תצפית קיצונית מאוד, לעיתים החציון לא ישתנה כלל — בניגוד לממוצע.

\subsubsection{טעויות נפוצות}
\begin{itemize}
\item לשכוח לסדר את הנתונים לפני חישוב.
\item לחשוב שהחציון “חייב להיות” אחד הערכים בהתפלגות (לא תמיד).
\end{itemize}



\subsection{הממוצע}

\subsubsection{הגדרה}
\textbf{הממוצע} הוא סכום כל התצפיות חלקי מספר התצפיות:
\[
\bar{X} = \frac{1}{n}\sum_{i=1}^{n} x_i
\]

\subsubsection{הסבר אינטואיטיבי}
הממוצע הוא נקודת “שיווי משקל” של ההתפלגות.

\subsubsection{דגשים למבחן}
\begin{itemize}
\item הממוצע מתאים לסולם רווח ומעלה.
\item הממוצע \textbf{רגיש לערכים קיצוניים}.
\item הוספת תצפית:
  \begin{itemize}
  \item גדולה מהממוצע → הממוצע יגדל.
  \item קטנה מהממוצע → הממוצע יקטן.
  \item שווה לממוצע → הממוצע לא ישתנה.
  \end{itemize}
\end{itemize}

\subsubsection{דוגמה קצרה}
הורדת תצפית נמוכה מהממוצע תגרום לממוצע לעלות.

\subsubsection{טעויות נפוצות}
\begin{itemize}
\item לפרש ממוצע כערך “טיפוסי” גם בהתפלגות מוטה.
\item להשתמש בממוצע כשאין משמעות לרווחים (סולם סדר).
\end{itemize}



\subsection{השוואה בין מדדי הנטייה המרכזית}

\begin{center}
\begin{tabular}{|c|c|c|c|}
\hline
מדד & סולם מינימלי & רגישות לקיצוניים & תמיד קיים \\
\hline
שכיח & שמי & לא & לא \\
חציון & סדר & נמוכה & כן \\
ממוצע & רווח & גבוהה & כן \\
\hline
\end{tabular}
\end{center}



\subsection{פונקציות הפסד וקשר למדדי מרכז}

\subsubsection{רעיון כללי}
פונקציית הפסד מודדת “כמה טעינו” כאשר אנו מייצגים את ההתפלגות בערך אחד.

\subsubsection{קשרים חשובים}
\begin{itemize}
\item מזעור סכום \textbf{ריבועי הסטיות}:
\[
\sum (x_i - c)^2 \Rightarrow c = \bar{X}
\]
הממוצע.
\item מזעור סכום \textbf{הסטיות המוחלטות}:
\[
\sum |x_i - c| \Rightarrow c = \text{חציון}
\]
\item מזעור מספר הטעויות (0/1):
\[
\Rightarrow c = \text{שכיח}
\]
\end{itemize}

\subsubsection{דגש למבחן}
שאלות על פונקציות הפסד הן דרך עקיפה לשאול:  
\textbf{איזה מדד מרכז מתאים כאן?}



\subsection{טעויות נפוצות ובחינות קלאסיות}

\begin{itemize}
\item לחשב מדד מרכז “טכני” בלי לבדוק התאמה לסולם המדידה.
\item להתעלם מערכים קיצוניים כששואלים על ממוצע.
\item לשכוח שחציון ושכיח לא בהכרח משתנים כשנתון בודד משתנה.
\end{itemize}

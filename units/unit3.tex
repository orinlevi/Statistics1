\section{יחידה 3: מדדי פיזור}

מדדי פיזור מתארים עד כמה התצפיות שונות זו מזו ועד כמה הן מפוזרות סביב ערך מרכזי.
שני מדגמים יכולים להיות בעלי אותו מדד מרכז אך פיזור שונה מאוד.

\subsection{מהו פיזור ולמה צריך אותו}

\subsubsection{הסבר אינטואיטיבי}
פיזור מתאר את מידת ההטרוגניות של הנתונים:
\begin{itemize}
\item פיזור גדול → הנתונים רחוקים זה מזה.
\item פיזור קטן → הנתונים מרוכזים סביב ערך מסוים.
\end{itemize}

\subsubsection{דגשים למבחן}
\begin{itemize}
\item מדדי פיזור תמיד אי־שליליים.
\item מדדי פיזור נבחנים תמיד ביחס לערך מרכזי כלשהו.
\end{itemize}



\subsection{אחוז טעויות \texorpdfstring{\textenglish{(Ratio of Variation)}}{(Ratio of Variation)}}

\subsubsection{הגדרה}
אחוז התצפיות השונות מהשכיח:
\[
V = \frac{n - f_{Mo}}{n} \cdot 100
\]

כאשר $f_{Mo}$ היא שכיחות השכיח.

\subsubsection{הסבר אינטואיטיבי}
מודד כמה מהנתונים \emph{לא} שווים לערך הנפוץ ביותר.

\subsubsection{דגשים למבחן}
\begin{itemize}
\item בן הזוג של \textbf{השכיח}.
\item מתאים במיוחד לסולם שמי.
\item ערך גבוה → פיזור גדול.
\end{itemize}

\subsubsection{טעויות נפוצות}
\begin{itemize}
\item לבלבל בין אחוז טעויות לבין סטיית תקן.
\end{itemize}



\subsection{ממוצע סטיות מוחלטות (MAD)}

\subsubsection{הגדרה}
ממוצע המרחקים המוחלטים מהחציון:
\[
MAD = \frac{1}{n}\sum_{i=1}^{n} |x_i - \text{Median}|
\]

\subsubsection{הסבר אינטואיטיבי}
כמה בממוצע כל תצפית רחוקה מהחציון.

\subsubsection{דגשים למבחן}
\begin{itemize}
\item בן הזוג של \textbf{החציון}.
\item עמיד יחסית לערכים קיצוניים.
\item מתאים לפחות לסולם סדר.
\end{itemize}

\subsubsection{טרנספורמציות}
\begin{itemize}
\item הוספת קבוע לכל הערכים → MAD לא משתנה.
\item הכפלה בקבוע $b$ → MAD מוכפל ב-$|b|$.
\item שינוי סימן (כפל ב־$-1$) → MAD לא משתנה.
\end{itemize}

\subsubsection{טעויות נפוצות}
\begin{itemize}
\item לחשוב ש-MAD יכול להיות שלילי (לא נכון).
\end{itemize}



\subsection{שונות מדגמית}

\subsubsection{הגדרה}
ממוצע ריבועי הסטיות מהממוצע:
\[
S_n^2 = \frac{1}{n}\sum_{i=1}^{n} (x_i - \bar{X})^2
\]


\subsubsection{הסבר אינטואיטיבי}
מודד פיזור דרך ריבוע המרחקים מהממוצע – מדגיש ערכים קיצוניים.

\subsubsection{אינטואיציה: למה הממוצע ממזער את השונות}

השונות מבוססת על \textbf{ריבוע הסטיות מהממוצע}. ריבוע הסטייה גורם לכך שסטיות גדולות
“נענשות” הרבה יותר מאשר סטיות קטנות.

כאשר מזיזים את ערך המרכז מעט ימינה או שמאלה:
\begin{itemize}
\item חלק מהסטיות קטנות.
\item אחרות גדלות.
\end{itemize}

אולם, מכיוון שההפסד הוא ריבועי, ההגדלה של סטיות שכבר היו גדולות שוקלת יותר
מההקטנה של סטיות קטנות. לכן כל הזזה מנקודת האיזון מגדילה את סכום ריבועי הסטיות.

נקודת האיזון מתקבלת כאשר הסטיות “מושכות” באופן שווה לשני הכיוונים, כלומר כאשר
סכום הסטיות הוא אפס. נקודה זו היא בדיוק \textbf{הממוצע}.

במילים אחרות: הממוצע הוא נקודת שיווי־המשקל שממזערת את סכום ריבועי הסטיות.

\subsubsection{דגשים למבחן}
\begin{itemize}
\item בן הזוג של \textbf{הממוצע}.
\item רגישה מאוד לערכים קיצוניים.
\item תמיד אי־שלילית.
\item שונות שווה ל־0 רק כאשר כל הערכים זהים.
\end{itemize}

\subsubsection{יחידות מדידה}
השונות נמדדת ב\textbf{יחידות בריבוע} (למשל: ס"מ$^2$).

\subsubsection{הערה על חלוקה ב-$n$ לעומת $n-1$ (תיקון בסל)}
בקורסים שונים משתמשים בשתי הגדרות נפוצות לשונות במדגם:
\[
S_n^2=\frac{1}{n}\sum_{i=1}^n (x_i-\bar{X})^2
\qquad\text{או}\qquad
S^2=\frac{1}{n-1}\sum_{i=1}^n (x_i-\bar{X})^2
\]
החלוקה ב-$n-1$ נותנת אומד \textbf{חסר הטיה} לשונות האוכלוסייה (כשדוגמים מאוכלוסייה),
ואילו החלוקה ב-$n$ נפוצה בשימוש \textbf{תיאורי} למדגם עצמו.
בפתרון תרגילים יש לעבוד לפי הנוסחאות הנהוגות בקורס ולשמור עקביות בכל החישובים.

\subsection{סטיית תקן}

\subsubsection{הגדרה}
שורש השונות:
\[
S_n = \sqrt{S_n^2}
\]

\subsubsection{הסבר אינטואיטיבי}
מרחק “טיפוסי” של תצפית מהממוצע.

\subsubsection{דגשים למבחן}
\begin{itemize}
\item נמדדת באותן יחידות של המשתנה.
\item רגישה לערכים קיצוניים.
\end{itemize}



\subsection{השפעת טרנספורמציות על שונות וסטיית תקן}

\begin{itemize}
\item הוספת קבוע $a$ לכל הערכים:
  \begin{itemize}
  \item ממוצע ↑ ב-$a$
  \item שונות וסטיית תקן – לא משתנות
  \end{itemize}
\item הכפלה בקבוע $b$:
  \begin{itemize}
  \item ממוצע מוכפל ב-$b$
  \item שונות מוכפלת ב-$b^2$
  \item סטיית תקן מוכפלת ב-$|b|$
  \end{itemize}
\end{itemize}



\subsection{טווח (Range)}

\subsubsection{הגדרה}
הפרש בין הערך המקסימלי למינימלי:
\[
Range = \max - \min
\]

\subsubsection{דגשים למבחן}
\begin{itemize}
\item מושפע מאוד מערכים קיצוניים.
\item לא מתייחס לערך מרכזי.
\end{itemize}



\subsection{טווח בין־רבעוני (IQR)}

\subsubsection{הגדרה}
\[
IQR = Q_3 - Q_1
\]

\subsubsection{הסבר אינטואיטיבי}
מכיל את 50\% האמצעיים של ההתפלגות.

\subsubsection{דגשים למבחן}
\begin{itemize}
\item עמיד לערכים קיצוניים.
\item מבוסס על חציון ורבעונים.
\item מתאים במיוחד להתפלגויות מוטות.
\end{itemize}



\subsection{בחירת מדדי פיזור – סיכום מבחני}

\begin{itemize}
\item סולם שמי → שכיח + אחוז טעויות
\item סולם סדר / התפלגות עם קיצוניים → חציון + MAD / IQR
\item סולם רווח / יחס, התפלגות סימטרית → ממוצע + שונות / סטיית תקן
\end{itemize}



\subsection{טעויות נפוצות במבחנים}

\begin{itemize}
\item לחשב שונות בלי להבין השפעת טרנספורמציה.
\item להשוות שונות בין משתנים שנמדדים ביחידות שונות.
\item לשכוח ששונות נמדדת ביחידות בריבוע.
\item להניח שפיזור קטן אומר “אין שונות”.
\end{itemize}

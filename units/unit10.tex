\section{יחידה 10: התפלגות הדגימה}

יחידה זו מהווה מעבר מסטטיסטיקה תיאורית לסטטיסטיקה היסקית.
המטרה היא להבין כיצד ניתן להסיק על אוכלוסייה שלמה מתוך מדגם,
באמצעות התפלגות תיאורטית של סטטיסטי.



\subsection{המוטיבציה: למה צריך התפלגות דגימה}

במחקר אמיתי:
\begin{itemize}
\item האוכלוסייה גדולה מאוד או אינסופית
\item לא ניתן למדוד את כולה
\item לכן משתמשים במדגם
\end{itemize}

השאלה המרכזית:
\begin{quote}
עד כמה ממוצע מדגם מסוים הוא תוצאה סבירה של דגימה מאוכלוסייה נתונה?
\end{quote}



\subsection{הבעיה בהשוואה לאנשים בודדים}

השוואה של ממוצע מדגם להתפלגות של תצפיות בודדות היא \textbf{לא הוגנת}:
\begin{itemize}
\item ממוצע של מדגם יציב יותר מערך בודד
\item ככל ש־$n$ גדל, קשה יותר לקבל ערכים קיצוניים
\end{itemize}

לכן נדרש בסיס השוואה חדש.



\subsection{התפלגות הדגימה – הגדרה}

\textbf{התפלגות הדגימה} היא התפלגות תיאורטית של סטטיסטי מסוים,
הנבנית מאינסוף מדגמים מקריים בגודל $n$ מאותה אוכלוסייה.

\subsubsection{חשוב}
כל התפלגות דגימה מוגדרת ביחס ל:
\begin{itemize}
\item סטטיסטי מסוים (למשל: ממוצע)
\item גודל מדגם מסוים ($n$)
\end{itemize}



\subsection{התפלגות הדגימה של הממוצעים}

זוהי התפלגות של ממוצעי מדגמים, ולא של תצפיות בודדות.

\subsubsection{בניית ההתפלגות (רעיונית)}
\begin{enumerate}
\item דוגמים מדגם בגודל $n$
\item מחשבים ממוצע
\item מחזירים לאוכלוסייה
\item חוזרים על התהליך אינסוף פעמים
\end{enumerate}

\subsubsection{מה מייצג $X_i$}

$X_i$ מייצג תצפית אחת אקראית מהאוכלוסייה.

כלומר:
\begin{itemize}
\item $X_1$ – התצפית הראשונה במדגם
\item $X_2$ – התצפית השנייה במדגם
\item $\vdots$
\item $X_n$ – התצפית ה־$n$
\end{itemize}

כל $X_i$ מתפלג כמו המשתנה באוכלוסייה,
ולכולם אותה תוחלת ואותה שונות.

\subsection{פרמטרים של התפלגות הדגימה}

\subsubsection{התוחלת}
\[
\mu_{\bar{X}} = \mu
\]

תוחלת התפלגות הדגימה של הממוצעים שווה לתוחלת האוכלוסייה.

\subsubsection{סטיית התקן – טעות התקן}
\[
\sigma_{\bar{X}} = \frac{\sigma}{\sqrt{n}}
\]

\subsubsection{הנחת אי־תלות}

הנוסחאות עבור התוחלת והשונות של ממוצע המדגם
נשענות על ההנחה כי:
\[
X_1, \dots, X_n \text{ בלתי־תלויים}
\]

במקרה זה:
\[
Cov(X_i, X_j) = 0 \quad \text{לכל } i \neq j
\]

ללא אי־תלות, נוסחת השונות של ממוצע המדגם אינה תקפה.

\begin{itemize}
\item טעות התקן קטנה מסטיית התקן של האוכלוסייה,
משום שממוצעי מדגמים מתפזרים פחות מערכים בודדים.
\item ככל שגודל המדגם $n$ גדל,
טעות התקן קטנה,
והתפלגות הדגימה נעשית צרה יותר סביב $\mu$.
\end{itemize}

\subsubsection{למה הפרמטרים האלה נכונים}

אם $X_1, \dots, X_n$ הם משתנים מקריים בלתי־תלויים עם:
\[
E[X_i] = \mu \qquad , \qquad Var(X_i) = \sigma^2
\]

ונגדיר את ממוצע המדגם:
\[
\bar{X} = \frac{1}{n}\sum_{i=1}^n X_i
\]

אז מתקיים:
\[
E[\bar{X}] = \mu
\qquad\text{ו}\qquad
Var(\bar{X}) = \frac{\sigma^2}{n}
\]

ולכן:
\[
\sigma_{\bar{X}} = \frac{\sigma}{\sqrt{n}}
\]

כלומר:
הממוצע של ממוצעי המדגמים שווה לממוצע האוכלוסייה,
והפיזור קטן ככל שגודל המדגם גדל.

\subsubsection{מדוע שונות של סכום מתפרקת}

כאשר מחשבים שונות של ממוצע מדגם,
יש צורך לחשב שונות של סכום משתנים מקריים.

באופן כללי:
\[
Var(X_1 + \dots + X_n)
=
\sum_{i=1}^n Var(X_i)
\;+\;
2\sum_{i<j} Cov(X_i, X_j)
\]

במקרה של דגימה מקרית מהאוכלוסייה,
המשתנים $X_1,\dots,X_n$ הם \textbf{בלתי־תלויים},
ולכן:
\[
Cov(X_i, X_j) = 0 \quad \text{לכל } i \neq j
\]

כתוצאה מכך, איברי הצלב נעלמים,
ומתקבל:
\[
Var(X_1 + \dots + X_n)
=
\sum_{i=1}^n Var(X_i)
\]

פירוק זה תקף \textbf{רק} תחת הנחת אי־תלות.

\subsubsection{תקציר אינטואיטיבי}

\begin{itemize}
\item מעבר מממוצע רגיל לתוחלת:
כאשר מדגמים באקראי מאוכלוסייה,
הממוצע המדגמי מתכנס לתוחלת האוכלוסייה.
התוחלת היא הממוצע התאורטי של תהליך דגימה אינסופי.

\item למה $E(X_i)=E(X)$:
כל $X_i$ הוא תצפית אקראית מאותה אוכלוסייה,
ולכן יש לו אותה התפלגות, אותה תוחלת ואותה שונות.

\item למה $E(\bar X)=\mu$:
תוחלת היא אופרטור לינארי,
ולכן תוחלת של סכום היא סכום התוחלות,
גם ללא תלות בין המשתנים.

\item למה השונות קטנה ב-$n$:
שונות מודדת פיזור.
כאשר מחשבים ממוצע של $n$ משתנים בלתי־תלויים,
הסטיות נוטות להתאזן זו עם זו,
ולכן הפיזור קטן פי $n$.

\item תפקיד אי־התלות:
אי־תלות מבטיחה שאין קו־וריאנס בין $X_i$ ל-$X_j$,
ולכן אין איברי צלב בשונות של הממוצע.
ללא אי־תלות — נוסחת טעות התקן אינה תקפה.

\item ההבדל בין ערך בודד לממוצע מדגם:
ערך בודד מושווה לסטיית התקן $\sigma$,
בעוד שממוצע מדגם מושווה לטעות התקן $\sigma/\sqrt{n}$.
לכן אותו פער מספרי יכול להיראות קיצוני מאוד כערך בודד,
אך סביר כממוצע מדגם.

\item למה מותר להשתמש ב־Z:
כאשר התפלגות הדגימה נורמלית (או בקירוב),
ניתן לתקנן את ממוצע המדגם ולהשתמש בטבלת Z
כדי להעריך הסתברויות.
\end{itemize}

\subsection{משמעות טעות התקן}

טעות התקן מודדת את מידת הפיזור של ממוצעי מדגמים סביב תוחלת האוכלוסייה.

\subsubsection{דגש למבחן}
ככל ש־$n$ גדול יותר:
\begin{itemize}
\item התפלגות הדגימה צרה יותר
\item ממוצע מדגם קיצוני נעשה פחות סביר
\end{itemize}



\subsection{צורת התפלגות הדגימה}

\subsubsection{מקרה מיוחד: אוכלוסייה נורמלית}

אם המשתנה באוכלוסייה מתפלג נורמלית:
\[
X \sim N(\mu,\sigma^2)
\]

אז לכל גודל מדגם $n$, ממוצע המדגם מתפלג נורמלית:
\[
\bar X \sim N\!\left(\mu,\frac{\sigma^2}{n}\right)
\]

תוצאה זו נובעת מהעובדה שסכום של משתנים מקריים נורמליים
בלתי־תלויים הוא עצמו משתנה מקרי נורמלי,
כאשר התוחלת היא סכום התוחלות והשונות היא סכום השונויות.

ההוכחה הפורמלית של טענה זו אינה טריוויאלית,
ומשתמשת בכלים מתמטיים מתקדמים כגון פונקציות אופי
או פונקציות יוצרות מומנטים,
ולכן אינה נכללת במסגרת הקורס.

\subsubsection{משפט הגבול המרכזי}
כאשר $n \ge 30$, התפלגות הדגימה של הממוצעים תהיה נורמלית בקירוב,
ללא קשר לצורת ההתפלגות באוכלוסייה.

\subsubsection{מקרה מיוחד}
אם המשתנה מתפלג נורמלית באוכלוסייה:
\begin{itemize}
\item התפלגות הדגימה נורמלית \textbf{לכל} $n$
\end{itemize}

\subsubsection{הערה על הוכחת משפט הגבול המרכזי}

משפט הגבול המרכזי הוא אחת התוצאות העמוקות והחשובות ביותר בהסתברות,
אך ההוכחה שלו אינה טריוויאלית ואינה נכללת במסגרת הקורס.

המשפט קובע כי סכום (או ממוצע) של משתנים מקריים בלתי־תלויים
ובעלי תוחלת ושונות סופיות,
מתכנס בהתפלגותו להתפלגות נורמלית,
גם כאשר ההתפלגות באוכלוסייה אינה נורמלית.

ההוכחות הפורמליות למשפט משתמשות בכלים מתקדמים כגון:
\begin{itemize}
\item פונקציות אופי (Characteristic Functions)
\item טרנספורם פורייה
\item משפטי התכנסות של פונקציות
\item גבולות חלשים של התפלגויות
\end{itemize}

הקורס עושה שימוש במשפט כתוצאה נתונה,
ומסתפק בפרשנות האינטואיטיבית שלו:
כאשר גודל המדגם גדול,
השפעת הצורה המקורית של ההתפלגות באוכלוסייה נחלשת,
והממוצע המדגמי מתנהג בקירוב כמו משתנה נורמלי.

\subsubsection{למה $n \ge 30$}

המספר 30 הוא כלל אצבע ולא גבול חד.

בפועל:
\begin{itemize}
\item בהתפלגויות סימטריות – לעיתים מספיק $n$ קטן יותר
\item בהתפלגויות מוטות או קיצוניות – נדרש $n$ גדול יותר
\end{itemize}

\subsubsection{מקרי קצה של גודל המדגם}

\paragraph{המקרה $n=1$}
כאשר גודל המדגם הוא $n=1$ מתקבל:
\[
\bar X = X
\qquad\Rightarrow\qquad
\sigma_{\bar X} = \sigma
\]

כלומר:
אין הבדל בין תצפית בודדת לבין ממוצע מדגם,
והתפלגות הדגימה חופפת להתפלגות האוכלוסייה.

\paragraph{המקרה $n \to \infty$}
כאשר גודל המדגם שואף לאינסוף:
\[
\sigma_{\bar X} = \frac{\sigma}{\sqrt{n}} \to 0
\]

במקרה זה ממוצעי המדגמים מתכנסים לתוחלת האוכלוסייה,
וההשפעה של מקריות הדגימה נעלמת.

מכאן נובע שככל ש־$n$ גדול יותר,
הממוצע המדגמי יציב ומדויק יותר.


\subsection{דוגמה אינטואיטיבית: אחוזון באוכלוסייה לעומת ממוצע מדגם}

נניח כי המשתנה באוכלוסייה מתפלג נורמלית, וכי הערך $44$ הוא \textbf{האחוזון ה־44} באוכלוסייה.
כלומר:
\[
P(X \le 44) = 0.44
\]
מאחר שהאוכלוסייה נורמלית, ערך זה נמצא משמאל לתוחלת $\mu$ של האוכלוסייה.

כעת נבחר מדגם מקרי בגודל $n=44$ ונבחן את ההסתברות ש\textbf{ממוצע המדגם} יהיה גדול מ־$44$:
\[
P(\bar X > 44)
\]

התפלגות הדגימה של ממוצע המדגם מתפלגת נורמלית סביב אותה תוחלת $\mu$,
אך עם סטיית תקן קטנה יותר:
\[
\sigma_{\bar X} = \frac{\sigma}{\sqrt{n}}
\]

כתוצאה מכך, הערך $44$ נעשה \textbf{רחוק יותר מן המרכז במונחי סטיות תקן},
והשטח שמימין לו בהתפלגות הדגימה גדל משמעותית
לעומת השטח שמימין לו בהתפלגות האוכלוסייה.

בדוגמה זו מתקבל בקירוב:
\[
P(\bar X > 44) \approx 0.84
\]

\subsubsection{המחשה גרפית מדויקת}

\begin{center}
\begin{english}
\begin{tikzpicture}
\begin{axis}[
    no markers,
    domain=30:80,
    samples=200,
    ymin=0, ymax=0.16, % גובה שמאפשר לגבעה הצרה להתנשא מעל הרחבה
    axis lines=left,
    width=0.85\textwidth,
    height=6cm,
    xlabel={Value},
    ylabel={Density},
    xlabel style={at={(ticklabel* cs:1)}, anchor=north west},
    ylabel style={at={(ticklabel* cs:1)}, anchor=south west},
    tick label style={font=\small},
    legend style={at={(0.95,0.95)}, anchor=north east, font=\small},
    xtick={44, 55},
    xticklabels={44, $\mu$},
    enlargelimits=false,
    clip=true,
    axis on top,
    grid = major,
    grid style={dashed, gray!20}
]

% פונקציית צפיפות נורמלית
\def\gauss#1#2{1/(#2*sqrt(2*pi))*exp(-((x-#1)^2)/(2*#2^2))}

% ממוצע משותף לשתי ההתפלגויות
\def\mval{55}

% 1) אוכלוסייה: רחבה ונמוכה (סטיית תקן = 10)
\addplot [name path=pop, blue, thick] {\gauss{\mval}{10}};
\addlegendentry{Population ($n=1$)}

% 2) מדגם: צרה וגבוהה (סטיית תקן = 3, המייצגת 10/sqrt(11) בקירוב)
\addplot [name path=sample, red, thick] {\gauss{\mval}{3}};
\addlegendentry{Sample Mean ($n=44$)}

% ציר X למילוי שטחים
\path[name path=axis] (axis cs:30,0) -- (axis cs:80,0);

% צביעת השטח מימין ל-44 באוכלוסייה (56% מהשטח הכחול)
\addplot [fill=blue, fill opacity=0.1] fill between [of=pop and axis, soft clip={domain=44:80}];

% צביעת השטח מימין ל-44 במדגם (כמעט כל השטח האדום - כ-84%+)
\addplot [fill=red, fill opacity=0.3] fill between [of=sample and axis, soft clip={domain=44:80}];

% קו הפרדה בערך 44
\draw [dashed, thick, black!60] (axis cs:44,0) -- (axis cs:44,0.14) node[pos=0.9, left, font=\footnotesize] {Value: 44};

\end{axis}
\end{tikzpicture}
\end{english}
\end{center}

\textbf{שימי לב:} בחירה של \texttt{std\_pop=10} היא לצורכי המחשה בלבד;
הצורה היחסית (הצרות של התפלגות הממוצעים והגדלת השטח מימין ל־$44$)
אינה תלויה בערך המספרי המדויק של סטיית התקן.

\subsubsection{משפט זהב}

כאשר ערך נתון נמצא \textbf{מתחת לתוחלת האוכלוסייה},
הצרות של התפלגות ממוצע המדגם גורמת לעלייה חדה
בהסתברות שממוצע המדגם יהיה גדול מערך זה,
וככל ש־$n$ גדל — האפקט מתחזק.

\subsection{חישוב הסתברויות בהתפלגות הדגימה}

כאשר התפלגות הדגימה נורמלית, ניתן לחשב ציוני תקן:

\[
Z = \frac{\bar{X} - \mu}{\sigma / \sqrt{n}}
\]

ולתרגם אותם להסתברויות באמצעות טבלת Z.

\subsubsection{קשר להתפלגות הנורמלית הסטנדרטית}

לאחר תקנון:
\[
Z = \frac{\bar{X} - \mu}{\sigma / \sqrt{n}}
\]

המשתנה $Z$ מתפלג בקירוב:
\[
Z \sim N(0,1)
\]

ולכן ניתן להשתמש בטבלת Z כפי שנעשה ביחידה 9.

\subsection{מה מאפשרת התפלגות הדגימה}

\begin{itemize}
\item בדיקת מידת הקיצוניות של ממוצע מדגם
\item השוואה הוגנת בין מדגם לאוכלוסייה
\item בסיס לבדיקת השערות סטטיסטיות
\end{itemize}

\subsubsection{מה היא \textbf{לא} מאפשרת}
\begin{itemize}
\item חישוב תוחלת האוכלוסייה
\item הסקת צורת ההתפלגות באוכלוסייה
\end{itemize}

\subsection{אזהרה חשובה}

התפלגות הדגימה מתארת סטטיסטי (כגון ממוצע),
ולא תצפיות בודדות.

לכן:
\begin{itemize}
\item ערך בודד נמדד ביחס ל־$\sigma$
\item ממוצע מדגם נמדד ביחס ל־$\sigma/\sqrt{n}$
\end{itemize}

\subsection{טעויות נפוצות במבחן}

\begin{itemize}
\item לבלבל בין התפלגות האוכלוסייה להתפלגות הדגימה
\item לחשוב שהתפלגות הדגימה מורכבת ממספר סופי של מדגמים
\item לשכוח לחלק ב־$\sqrt{n}$ בעת חישוב טעות התקן
\item להשוות ממוצע מדגם להתפלגות של ערכים בודדים
\end{itemize}



\subsection{דגש מסכם ליחידה}

לפני חישוב, שאלי:
\begin{enumerate}
\item האם מדובר בממוצע מדגם?
\item מהו $n$?
\item האם התפלגות הדגימה נורמלית?
\item האם אני משתמשת בטעות התקן ולא בסטיית התקן של האוכלוסייה?
\end{enumerate}

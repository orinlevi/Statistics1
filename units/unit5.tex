\section{יחידה 5: קומבינטוריקה}

קומבינטוריקה עוסקת בספירת מספר האפשרויות לבחירה או לסידור של אובייקטים, בהתאם לכללי הבחירה:  
האם יש חשיבות לסדר? האם יש החזרה? האם יש תנאים או מגבלות?



\subsection{עקרון הכפל}

\subsubsection{הגדרה}
כאשר תהליך מורכב ממספר שלבים בלתי תלויים, ומספר האפשרויות בכל שלב ידוע –  
מספר האפשרויות הכולל הוא מכפלת האפשרויות בכל שלב.

\subsubsection{הסבר אינטואיטיבי}
אם יש 3 אפשרויות לשלב ראשון ו־5 לשלב שני, אז לכל בחירה ראשונה יש 5 המשכים.

\subsubsection{דוגמה}
בחירת קוד בן 2 תווים:
\begin{itemize}
\item תו ראשון: 4 אפשרויות
\item תו שני: 10 אפשרויות
\end{itemize}
סה״כ: $4 \cdot 10 = 40$

\subsubsection{דגש למבחן}
עובד רק אם כל שלב מתבצע \textbf{ללא תלות} בבחירות הקודמות.



\subsection{פרמוטציות – סידור עם חשיבות לסדר}

\subsubsection{פרמוטציה ללא חזרות}

\textbf{הגדרה}: סידור של $n$ איברים שונים כאשר הסדר חשוב.

\[
n! = n \cdot (n-1) \cdot \dots \cdot 1
\]

\textbf{דוגמה}: סידור 5 סטודנטים בשורה: $5!$

\subsubsection{דגשים למבחן}
\begin{itemize}
\item אם הסדר חשוב – זו פרמוטציה.
\item מילים כמו: \textbf{סידור, שורה, תור} → לרוב פרמוטציה.
\end{itemize}



\subsection{פרמוטציות עם חזרות}

\subsubsection{הגדרה}
כאשר מסדרים $n$ איברים, אך חלקם זהים זה לזה.

\[
\frac{n!}{k_1! \cdot k_2! \cdots}
\]

כאשר $k_i$ הוא מספר הפעמים שכל איבר זהה מופיע.

\subsubsection{דוגמה}
המילה ״תותים״ (5 אותיות, ת׳ פעמיים):
\[
\frac{5!}{2!}
\]

\subsubsection{דגש למבחן}
זהות ≠ חזרה.  
אם האיברים \textbf{זהים לחלוטין} – חייבים לחלק בפקטוריאל.



\subsection{וריאציות – בחירה עם חשיבות לסדר}

\subsubsection{ללא חזרות}
בחירה של $k$ איברים מתוך $n$, כאשר הסדר חשוב ואין חזרות:
\[
V(n,k) = \frac{n!}{(n-k)!}
\]

\subsubsection{עם חזרות}
בחירה של $k$ איברים מתוך $n$, כאשר הסדר חשוב ויש חזרות:
\[
n^k
\]

\subsubsection{דוגמאות}
\begin{itemize}
\item קוד בן 3 ספרות (עם חזרות): $10^3$
\item קוד בן 3 אותיות שונות: $\frac{26!}{23!}$
\end{itemize}

\subsubsection{דגש למבחן}
המילה \textbf{קוד} כמעט תמיד → יש חשיבות לסדר.



\subsection{קומבינציות – בחירה ללא חשיבות לסדר}
\subsubsection{שיטת המקלות והכוכבים}

שיטה לספירת מספר הדרכים לחלק מספר זהה של פריטים בין תאים שונים.

\subsubsection{מתי משתמשים}
\begin{itemize}
\item כאשר הפריטים זהים לחלוטין
\item כאשר הסדר לא חשוב
\item כאשר מותרת חזרה
\end{itemize}

\subsubsection{דוגמה}
כמה דרכים לחלק $k$ פריטים זהים בין $n$ תאים?

\subsubsection{הרעיון}
מייצגים:
\begin{itemize}
\item כוכבים (\(*\)) – הפריטים
\item מקלות (\(|\)) – הפרדה בין תאים
\end{itemize}

מספר הדרכים הוא:
\[
\binom{k+n-1}{n-1}
\]

\subsubsection{דגש למבחן}
אם כתוב:
\begin{quote}
״כמה דרכים לחלק…״
\end{quote}
וסדר לא חשוב → לחשוב מיד על מקלות וכוכבים.

\subsubsection{הגדרה}
בחירה של $k$ איברים מתוך $n$, כאשר הסדר \textbf{לא} חשוב ואין חזרות:
\[
\binom{n}{k} = \frac{n!}{k!(n-k)!}
\]

\subsubsection{הסבר אינטואיטיבי}
לא משנה באיזה סדר בחרנו – רק \textbf{מי נבחר}.

\subsubsection{דוגמאות}
\begin{itemize}
\item בחירת ועד
\item בחירת קבוצה
\item בחירת פריטים
\end{itemize}

\subsubsection{דגש למבחן}
אם שואלים רק \textbf{הרכב} ולא תפקידים – זו קומבינציה.



\subsection{״לפחות״, ״בדיוק״ ו־״לא יותר מ־״}

\subsubsection{לפחות}
לרוב נוח לעבוד עם המשלים:
\[
P(\text{לפחות אחד}) = 1 - P(\text{אף אחד})
\]

\subsubsection{בדיוק}
מחלקים למקרים שאינם חופפים ומסכמים.

\subsubsection{לא יותר מ־}
כולל כמה תרחישים → מחשבים כל אחד בנפרד ומחברים.

\subsubsection{דגש למבחן}
המילה \textbf{לפחות} היא נורת אזהרה – לעצור ולחשוב.



\subsection{שיבוץ עם מגבלות (רצפים אסורים)}

\subsubsection{עיקרון}
\begin{itemize}
\item סופרים את כל הסידורים
\item מחסירים סידורים לא רצויים
\item מוסיפים חיתוכים (עקרון ההכלה וההפרדה)
\end{itemize}

\subsubsection{טכניקת ה״גוש״}
כאשר שני איברים חייבים להיות סמוכים – מתייחסים אליהם כאל איבר אחד.

\subsubsection{דוגמה}
אם A ו־B חייבים להיות יחד:
\[
(A,B) \Rightarrow \text{גוש אחד}
\]



\subsection{טעויות נפוצות בקומבינטוריקה}

\begin{itemize}
\item להתבלבל בין סדר חשוב / לא חשוב
\item לשכוח לבדוק אם יש חזרות
\item לא לשים לב לזהות בין איברים
\item לשכוח להוסיף חיתוכים
\end{itemize}

\subsection{דגש מסכם למבחן}

לפני כל חישוב – לשאול:
\begin{enumerate}
\item סדר חשוב?
\item יש חזרות?
\item יש זהות?
\item יש תנאים?
\end{enumerate}




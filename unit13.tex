\section{יחידה 13: מעבר לבדיקת השערות – גודל אפקט ורווח סמך}

יחידה זו מסכמת את הקורס ומציגה הסתכלות ביקורתית על בדיקת השערות סטטיסטיות (NHST),
וכן כלים משלימים שמטרתם לספק תמונה מלאה ומשמעותית יותר של תוצאות מחקר:
\textbf{גודל אפקט} ו-\textbf{רווח סמך}.
הדגש עובר מהחלטה בינארית להבנת גודל האפקט ומידת אי־הוודאות.

\subsection{מגבלות בדיקת השערות סטטיסטיות (NHST)}

בדיקת השערות עונה על השאלה:
\begin{quote}
האם הנתונים סבירים בהנחה שהשערת האפס $H_0$ נכונה?
\end{quote}

אך לשיטה זו מגבלות מהותיות.

\subsubsection{תלות בגודל המדגם}

\begin{itemize}
\item כאשר $n$ גדול, טעות התקן קטנה.
\item לכן גם אפקטים קטנים מאוד עשויים להיות מובהקים סטטיסטית.
\item במדגמים קטנים, אפקטים גדולים עלולים לא לצאת מובהקים.
\end{itemize}

\subsubsection{מובהקות אינה חשיבות}

מובהקות סטטיסטית אינה מעידה בהכרח על חשיבות מעשית או קלינית של האפקט.

\subsubsection{שרירותיות רמת המובהקות}

\begin{itemize}
\item הבחירה ב-$\alpha = 0.05$ היא מוסכמה ולא חוק טבע.
\item ערכים קרובים לסף (למשל $p=0.049$ לעומת $p=0.051$)
עשויים להוביל למסקנות שונות לחלוטין.
\end{itemize}

\subsubsection{מסקנה בינארית}

בדיקת השערות מובילה להכרעה דיכוטומית:
\begin{itemize}
\item דוחים את $H_0$
\item או לא דוחים את $H_0$
\end{itemize}

ואינה מספקת מידע על גודל האפקט או אי־הוודאות באמידה.

\subsection{p-value ופרשנותו}

\textbf{p-value} מודד עד כמה הנתונים אינם סבירים בהנחה ש-$H_0$ נכונה.

\begin{itemize}
\item p-value אינו מודד את גודל האפקט.
\item p-value אינו הסתברות ש-$H_0$ נכונה.
\item ערך p קטן יותר מעיד על ראיות חזקות יותר נגד $H_0$,
אך לא בהכרח על אפקט גדול יותר.
\end{itemize}

\subsection{\hebeng{גודל אפקט}{Effect Size}}

\subsubsection{רעיון מרכזי}

גודל אפקט מודד \textbf{כמה גדול ההבדל או הקשר},
ולא רק האם הוא מובהק סטטיסטית.

\subsubsection{מאפיינים מרכזיים}

\begin{itemize}
\item מאפשר פרשנות מעשית של התוצאה
\item מאפשר השוואה בין מחקרים שונים
\item אינו תלוי בגודל המדגם בלבד
\end{itemize}

\subsubsection{\hebeng{מדד כהן $d$}{Cohen's d}}

\paragraph{מטרה}
מדד כהן $d$ מודד את \textbf{גודל ההבדל במונחי סטיות תקן}, כלומר גודל אפקט \textbf{מתוקנן} (Standardized),
כדי שאפשר יהיה לפרש ולהשוות בין מחקרים גם כשסקאלות שונות.

\paragraph{הגדרה (מקרה בסיסי: שתי קבוצות)}
כאשר משווים שני ממוצעים:
\[
d = \frac{\bar X_1 - \bar X_2}{s_{\text{pooled}}}
\]

כאשר $s_{\text{pooled}}$ היא סטיית תקן משוקללת (pooled):
\[
s_{\text{pooled}}=
\sqrt{\frac{(n_1-1)s_1^2 + (n_2-1)s_2^2}{n_1+n_2-2}}
\]

\paragraph{מקרה של מבחן חד-מדגמי מול ערך ייחוס}
כאשר משווים ממוצע מדגם לערך תיאורטי $\mu_0$:
\[
d = \frac{\bar X - \mu_0}{s}
\]

\paragraph{פרשנות אינטואיטיבית}
\begin{itemize}
\item $d$ אומר: \textbf{כמה סטיות תקן} מפרידים בין הממוצעים.
\item בניגוד ל-$p$-value, $d$ מתאר \textbf{חשיבות/עוצמה מעשית} (magnitude) ולא רק מובהקות.
\end{itemize}

\paragraph{כללי אצבע נפוצים (Cohen)}
\begin{itemize}
\item $d \approx 0.2$ אפקט קטן
\item $d \approx 0.5$ אפקט בינוני
\item $d \approx 0.8$ אפקט גדול
\end{itemize}

\paragraph{דגש למבחן}
ייתכן מצב של:
\begin{itemize}
\item $d$ קטן אך $p$ קטן מאוד (מדגם גדול)
\item $d$ גדול אך $p$ לא מובהק (מדגם קטן)
\end{itemize}
לכן בדיווח מלא רצוי לציין גם \textbf{מובהקות}, גם \textbf{גודל אפקט}, וגם \textbf{רווח סמך}.

\begin{figure}[H]
    \centering
    \begin{english}
    \begin{tcolorbox}[colback=blue!3, colframe=blue!30, width=0.9\textwidth, title=Cohen's $d$]
        \begin{hebrew}
        זהו המדד הסטנדרטי למדידת גודל האפקט ביחידות של סטיות תקן. 
        \begin{itemize}
            \item \textbf{אפקט קטן:} $d \approx 0.2$
            \item \textbf{אפקט בינוני:} $d \approx 0.5$
            \item \textbf{אפקט גדול:} $d \approx 0.8$
        \end{itemize}
        \end{hebrew}
    \end{tcolorbox}
    \end{english}
\end{figure}

\subsubsection{דגש למבחן}

ייתכנו המצבים:
\begin{itemize}
\item אפקט קטן אך מובהק (במדגם גדול)
\item אפקט גדול אך לא מובהק (במדגם קטן)
\end{itemize}

\subsection{רווח סמך (Confidence Interval)}

\subsubsection{הגדרה}

רווח סמך הוא תחום ערכים, המחושב על סמך מדגם,
אשר בשיטת בנייה נתונה מכיל את פרמטר האוכלוסייה
בהסתברות נתונה בטווח הארוך.

\subsubsection{רמת סמך}

רווח סמך של 95\% פירושו:
\begin{itemize}
\item אם נבנה רווחים רבים באותה שיטה
\item כ-95\% מהם יכילו את הפרמטר האמיתי
\end{itemize}

\textbf{אסור לפרש} זאת כהסתברות שהפרמטר נמצא בתוך הרווח.

\subsection{מבנה כללי של רווח סמך}

\[
\text{אומדן} \pm \text{שגיאה}
\]

כאשר השגיאה תלויה ב:
\begin{itemize}
\item רמת הסמך
\item טעות התקן
\item גודל המדגם
\end{itemize}

\subsection{\hebeng{רווח סמך לתוחלת (כאשר $\sigma$ ידועה)}{Confidence Interval for Mean (Known $\sigma$)}}
\[
\bar{X} \pm Z_{\alpha/2}\cdot \frac{\sigma}{\sqrt{n}}
\]

\begin{itemize}
\item מרכז הרווח הוא ממוצע המדגם
\item רוחב הרווח משקף את אי־הוודאות באמידה
\end{itemize}

\subsection{\hebeng{גורמים המשפיעים על רוחב רווח הסמך}{Factors Affecting CI Width}}

הרווח מוגדר כמרחק מהממוצע: $Z_{\alpha/2} \cdot \frac{\sigma_x}{\sqrt{n}}$.

\begin{itemize}
\item $n$ גדול יותר $\rightarrow$ רווח צר יותר
\item $\sigma$ גדולה יותר $\rightarrow$ רווח רחב יותר
\item רמת סמך גבוהה יותר $\rightarrow$ רווח רחב יותר
\end{itemize}

\begin{figure}[H]
    \centering
    \setlength{\fboxsep}{10pt} % המרוח בין התמונה למסגרת
    \setlength{\fboxrule}{1pt}  % עובי המסגרת
    \fcolorbox{black}{pink!14}{% רקע צהבהב להבלטת הגורמים המשפיעים
        \includegraphics[width=0.85\textwidth]{CI_impacts.png}
    }
    \caption{השפעת המשתנים על רווח בר הסמך}
\end{figure}

\begin{center}
\begin{minipage}{0.95\textwidth}
\begin{english}
\centering
\begin{tikzpicture}[node distance=1.5cm, auto]

    % המשתנים מהנוסחה
    \node (alpha) [statbox, fill=blue!5] {$\alpha \downarrow$ \\ (Confidence $\uparrow)$};
    \node (sigma) [statbox, right=0.5cm of alpha, fill=green!5] {$\sigma_x \uparrow$ \\ (Standard Deviation)};
    \node (n) [statbox, right=0.5cm of sigma, fill=red!5] {$n \uparrow$ \\ (Sample Size)};

    % התוצאה על הרווח
    \node (res_alpha) [below=1.2cm of alpha, font=\small] {Wider Interval $\uparrow$};
    \node (res_sigma) [below=1.2cm of sigma, font=\small] {Wider Interval $\uparrow$};
    \node (res_n) [below=1.2cm of n, font=\small] {Narrower Interval $\downarrow$};

    % חצים צבעוניים (לפי התמונה)
    \draw [stataxis, blue] (alpha) -- (res_alpha);
    \draw [stataxis, green!60!black] (sigma) -- (res_sigma);
    \draw [stataxis, red] (n) -- (res_n);

\end{tikzpicture}
\end{english}
\end{minipage}
\end{center}

\subsection{הקשר בין רווח סמך לבדיקת השערות}

קיים קשר ישיר בין:
\begin{itemize}
\item רווח סמך דו־צדדי ברמת סמך $1-\alpha$
\item בדיקת השערות דו־צדדית ברמת מובהקות $\alpha$
\end{itemize}

\begin{figure}[H]
    \centering
    \setlength{\fboxsep}{10pt} % המרחק בין התמונה למסגרת
    \setlength{\fboxrule}{1pt}  % עובי המסגרת
    \fcolorbox{black}{purple!5}{\includegraphics[width=0.85\textwidth]{typeIerror-power.png}}
    \caption{Probability of Rejecting the Null Hypothesis: Type I Error and Power}
\end{figure}

\begin{figure}[H]
    \centering
    \setlength{\fboxsep}{10pt}
    \fcolorbox{black}{purple!5}{\includegraphics[width=0.85\textwidth]{typeIIerror-confidence.png}}
    \caption{Probability of Non-Rejection of the Null Hypothesis: Type II Error and Confidence Level}
\end{figure}

\vspace{1cm}

\subsubsection{כלל חשוב}

\begin{itemize}
\item אם ערך לפי $H_0$ נמצא בתוך רווח הסמך $\rightarrow$ לא דוחים את $H_0$
\item אם ערך לפי $H_0$ מחוץ לרווח הסמך $\rightarrow$ דוחים את $H_0$
\end{itemize}

ההבדל הוא בייצוג:
בדיקת השערות מספקת החלטה בינארית,
ורווח סמך מספק טווח אי־ודאות רציף.

\subsection{שילוב הכלים}

בדיווח סטטיסטי מלא מומלץ לשלב:
\begin{itemize}
\item בדיקת השערות (מובהקות)
\item גודל אפקט
\item רווח סמך
\end{itemize}

שילוב זה מאפשר:
\begin{itemize}
\item קבלת החלטה
\item הבנת המשמעות המעשית
\item הערכת אי־הוודאות
\end{itemize}

\subsection{\hebeng{הקשר בין רווח סמך לעוצמת המבחן}{Confidence Intervals and Power}}

כפי שניתן לראות בתרשים, קיימת שרשרת של השפעות:
\begin{itemize}
    \item \textbf{צמצום רווח הסמך (CI):} נגרם לרוב על ידי הגדלת המדגם $(n)$ או הקטנת רמת הביטחון.
    \item \textbf{הגדלת $\alpha$:} גורמת לדחייה קלה יותר של $H_0$ (יותר שטח בגרף תחת אזור הדחייה).
    \item \textbf{שיפור עוצמת המבחן:} ככל שקל יותר לדחות את $H_0$, כך גדל הסיכוי שנזהה אפקט אמיתי $(1-\beta)$.
\end{itemize}

\vspace{1.5cm} % מוסיף רווח של סנטימטר וחצי בין התמונות

\begin{figure}[H]
    \centering
    \includegraphics[width=0.9\textwidth]{a_b_errors} % השתמשי בתמונה שכבר העלית
    \caption{המחשה גרפית של שטחי הטעות והעוצמה}
\end{figure}

\vspace{1.5cm} % מוסיף רווח של סנטימטר וחצי בין התמונות

\begin{center}
\begin{minipage}{0.95\textwidth}
\begin{english}
\centering
\begin{tikzpicture}[node distance=1.5cm, auto]

    % Nodes - English terms for clarity
    \node (n) [statbox, fill=purple!5] {Increase Sample Size \\ $(n)$};
    \node (ci) [statbox, below=of n, fill=blue!5] {Narrower Confidence \\ Interval (CI)};
    \node (reject) [statbox, below=of ci, fill=orange!5] {Easier to Reject \\ Null Hypothesis $(H_0)$};
    
    % Outcomes
    \node (alpha) [statbox, below left=1.8cm and 0.2cm of reject, fill=red!5] 
        {$\uparrow$ Type I Error \\ $(\alpha)$};
    \node (power) [statbox, below right=1.8cm and 0.2cm of reject, fill=green!5] 
        {$\uparrow$ Statistical Power \\ $(1-\beta)$};

    % Main Arrows
    \draw [stataxis] (n) -- (ci);
    \draw [stataxis] (ci) -- (reject);
    \draw [stataxis] (reject.south) -- ++(0,-0.5) -| (alpha.north);
    \draw [stataxis] (reject.south) -- ++(0,-0.5) -| (power.north);

    % Side Arrows to show direct influence
    \draw [stataxis, bend right=45] (n.west) to node[left, font=\tiny] {Direct Impact} (reject.west);
    \draw [stataxis, bend right=60] (n.west) to (alpha.west);

\end{tikzpicture}
\end{english}
\captionof{figure}{Detailed Flowchart: Sample Size Impact on Power and Error}
\end{minipage}
\end{center}

\subsection{דגש מסכם ליחידה}

בדיקת השערות מספקת קריטריון החלטה,
אך אינה מספיקה לבדה.

הסקה סטטיסטית מלאה דורשת:
\begin{itemize}
\item פרשנות מעבר ל-p-value
\item הערכת גודל האפקט
\item שימוש ברווחי סמך
\end{itemize}


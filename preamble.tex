
% ===============================
% שפה (עברית + English)
% ===============================
\usepackage{fontspec}
\usepackage{polyglossia}
%\setdefaultlanguage{hebrew}
\setmainlanguage{hebrew}
\setotherlanguage{english}
\newfontfamily\hebrewfont{DavidCLM-Medium}[
  Path = /Users/orinlevi/Library/Fonts/,
  Extension = .otf,
  BoldFont = DavidCLM-Bold,
  ItalicFont = DavidCLM-MediumItalic,
  BoldItalicFont = DavidCLM-BoldItalic,
]
\newfontfamily\hebrewfonttt{DavidCLM-Medium}[
  Path = /Users/orinlevi/Library/Fonts/,
  Extension = .otf,
]
% גופן לטיני לגרפים (TikZ/pgfplots) – מונע "nullfont" בתווים באנגלית (שם מערכת macOS/Windows)
\newfontfamily\figurelatinfont{Times New Roman}

% ===============================
% מתמטיקה
% ===============================
\usepackage{amsmath, amssymb}

% ===============================
% עמוד
% ===============================
\usepackage[a4paper,margin=2.5cm]{geometry}

% ===============================
% רשימות
% ===============================
\usepackage{enumitem}
\setlist[itemize]{itemsep=0.3em}

% ===============================
% טבלאות וצבעים (אם יש rowcolors וכד')
% ===============================
\usepackage[table,xcdraw]{xcolor}
\usepackage{longtable}

% ===============================
% גרפים וציורים (TikZ + pgfplots)
% ===============================
% --- חבילות גרפיקה וסימונים ---
\usepackage{float}       % לקיבוע תמונות ותרשימים עם [H]
\usepackage{caption}     % מאפשר שימוש ב-captionof מחוץ ל-figure
\usepackage{amssymb}     % לסימון V וסימנים מתמטיים
\usepackage{pifont}      % לסימון X (ding)
\usepackage{pgfplots}
\pgfplotsset{compat=1.18}
\usepgfplotslibrary{fillbetween}

\usepackage[most]{tcolorbox}

% --- הגדרות TikZ (גרסה חסינה) ---
\usepackage{tikz}
\usetikzlibrary{shapes.geometric, arrows.meta, positioning}

\tikzset{
        statbox/.style={
        rectangle, 
        rounded corners, 
        draw=black, 
        thick,
        fill=blue!5,
        text width=4.5cm, % הגדלנו מ-3.2cm ל-4.5cm כדי שיהיה רחב יותר
        minimum height=1.2cm, 
        text centered, 
        font=\footnotesize % הקטנו מעט את הגופן מ-small ל-footnotesize
    },
    stataxis/.style={thick, -{Stealth[scale=1.2]}}
}
% ===============================
% קישורים (אחרון)
% ===============================
\usepackage{hyperref}
\hypersetup{
    colorlinks=true,    % הופך את הטקסט עצמו לצבעוני (במקום מסגרת)
    linkcolor=blue,     % צבע קישורים פנימיים (כמו תוכן עניינים)
    citecolor=green,    % צבע הפניות למקורות/ביבליוגרפיה
    filecolor=magenta,  % צבע קישורים לקבצים
    urlcolor=cyan       % צבע קישורי אינטרנט חיצוניים
}